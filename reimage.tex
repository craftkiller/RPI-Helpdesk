%latex comments look like this
\documentclass[10pt]{article}
\special{papersize=8.5in,11in}
\usepackage[cm]{fullpage}
%\usepackage{titlesec}
%\titleformat{\section}{\large\bfseries}{\thesection}{1em}{}
%\titleformat{\subsection}{\normalsize\bfseries}{\thesection}{1em}{}
\begin{document}
\begin{flushright}
Re-image Instructions \today
\end{flushright}
\section{Copying the Image (must complete at the VCC)}
{\sc WARNING This will erase all data on your computer. Please back up your data.}
\begin{enumerate}
\item Please check the {\bf Image List} posted near Switch B for the version of Windows you want.
\item Plug in your laptop and connect it to the networking switch labeled ``Switch B''. \emph{\bf Do not use a retractable Ethernet cord.}
\item Turn on the laptop and press \emph{F12} immediately when the ThinkPad screen appears
\item Insert the re-image CD and select the option with the words ``CD'' or ``ATAPI''
\item If asked for operating system you want to install, type in either ``xp'', ``w7'', or ``w764bit''. and press \emph{Enter}. If you're not sure which to chose, ask a consultant.
\begin{description}
\item[w7:] This is a 32-bit version of windows. {\bf This is only offered as a BASE image.} Use this if you need to run 16-bit code.
\item[w764bit:] {\bf Recommended} This is a 64-bit version of windows. This version is offered as either a {\bf full} or {\bf base} image.
\end{description}
\item If asked for image type, type in either ``base'' or ``full'' and press \emph{Enter}. If no option is displayed, the base image is the only one available.
\begin{description}
\item[Base:] Operating System, Drivers, Antivirus, and Microsoft Office
\item[Full:] Base + all software included in mobile computing package
\end{description}
\item A blue screen with a progress bar will appear. When the progress bar reaches 100\%, follow the on-screen instructions and press ``Okay''. Your computer will shutdown on its own, albeit slowly. It will not restart on its own.
\item Return the CD and Network Cable to the helpdesk consultants. Don't forget to retrieve your RPI ID.
%\item If you are using a T60 or newer and you have installed {\bf XP}, you \emph{must} do the following (Otherwise skip to \emph{Updates}):
%\begin{enumerate} %begin T60 specific instructions
%\item Turn on the computer, and immediately press \emph{F1} at the ThinkPad screen. This will bring up the BIOS %options.
%\item Hit enter to choose the ``Config'' option, and then choose ``Serial ATA (SATA)''
%\item Press enter to select the mode: {\bf AHCI} for Windows 7, {\bf Compatibility} for XP
%\item Press the \emph{F10} key, press \emph{Enter} at confirmation, and the computer will restart.
%\end{enumerate} %end T60 specific instructions
\end{enumerate}

%\section{Initial set-up (should complete in the VCC)}
%\subsection{For Vista and Windows 7 Users:}
%\begin{enumerate}
%\item Agree to the term.
%\item Choose a username and password.
%\item Set the date.
%\item Unselect the ``Peace of Mind'' option.
%\end{enumerate}

%\subsection{For XP Users:}
%\begin{enumerate}
%\item Agree to the terms.
%\item Set your time-zone and language.
%\item Enter your name and organization (whatever you want).
%\item Enter a computer name, and choose an admin password. Do not forget this password.
%\item Continue hitting next until prompted for a Workground, and type in ``RENSSELAER''.
%\item Once logged in, create a new user account by going to Start $\rightarrow$ Control Panel $\rightarrow$ User Accounts $\rightarrow$ Create a new account. At this point, log off, and log back in with the new account.
%\end{enumerate}

\section{Updates (can complete at home, but please return these instructions)}
\begin{enumerate}
\item Disconnect from Switch B. Update your antivirus software by double-clicking the Microsoft System Center Endpoint Protection icon (a green or orange square), selecting the Update tab, and clicking the Update button.
\item Run Windows Updates by clicking {\bf Start $\rightarrow$ All Programs $\rightarrow$ Windows Update.}
\end{enumerate}
\subsection{{\sc optional} Enable Aero Theme for Windows 7}
By default, the image may disable visual effects, making Windows 7 look like older Windows releases. To enable Aero:
Right click on the {\bf Desktop}, click {\bf Personalize} and select a theme in the {\bf Aero Themes} section.
\subsection{{\sc optional} Enable Touchpad}
By default, the touchpad may be disabled. To enable it, click {\bf Start $\rightarrow$ Control Panel $\rightarrow$ Mouse $\rightarrow$ UltraNav} , and select checkboxes to match your preference.
\subsection{{\sc optional} Enable Wireless}
By default, the wireless may be disabled. To enable it, press \emph{Fn + F5} and enable the {\bf 802.11 Wireless Radio}. Also check the small wireless switch on your laptop.
%\begin{enumerate}
%\item Right click on the desktop
%\item go to Personalize at the bottom of the list.
%\item Select a new theme like \emph{Windows 7 Basic}
%\end{enumerate}
\section{Please return these instructions to the helpdesk!}
\emph{Unless these instructions are taped down, in which case, please leave them undisturbed.}
\end{document}
